\documentclass[lualatex,a4j,12pt]{mybook-ja}
  % 数学
    % すべての数式をdisplaystyleで表示
    %\everymath{\displaystyle}
  % 前付と後付の目次レベル
    %\renewcommand{\frontmattertoclevel}{part}
    %\renewcommand{\backmattertoclevel}{part}
  % 索引
    %\makeindex
  % ハイパーリンク
    \hypersetup{%
      % ページ数にリンクをつけない(テキストのみにリンクをつける)
      %linktoc=section,%
      % リンクに枠をつける(印刷時に残らない)
      %colorlinks=false,%
      % リンクの色をすべて変更
      %allcolors=blue,%
      % リンクに色や枠をつけない
      %hidelinks=true,%
      % PDFのブックマークを展開せず表示
      %bookmarksopen=false,%
    }
  % 省略記法
\begin{document}
  % レイアウト確認ページ
    %\layout
  % タイトル
    \renewcommand{\thepage}{C\arabic{page}}
    \thispagestyle{empty}
    \begin{titlepage}
      \title{\Huge\bfseries }
      \author[1]{}
      \affil[1]{}
      \date{\today}
      \maketitle
    \end{titlepage}
  % 目次
    \frontmatter
    \phantomsection
    \addcontentsline{toc}{\frontmattertoclevel}{\contentsname}
    \tableofcontents
  % 本文
    \mainmatter
    
  % 付録
    %\part*{\appendixpagename}
    %\appendix
    %\titleformat{\chapter}[display]{\Huge\bfseries}{\LARGE \appendixname \thechapter}{0em}{}
  % 参考文献
    %\backmatter
    %\phantomsection
    %\addcontentsline{toc}{\backmattertoclevel}{\bibname}
    %\bibliography{book-ja,book-en,paper}
  % 索引
    %\cleardoublepage
    %\phantomsection
    %\addcontentsline{toc}{\backmattertoclevel}{\indexname}
    %\printindex
\end{document}